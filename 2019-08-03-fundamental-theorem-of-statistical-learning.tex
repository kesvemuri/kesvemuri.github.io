\documentclass[]{article}
\usepackage{lmodern}
\usepackage{amssymb,amsmath}
\usepackage{ifxetex,ifluatex}
\usepackage{fixltx2e} % provides \textsubscript
\ifnum 0\ifxetex 1\fi\ifluatex 1\fi=0 % if pdftex
  \usepackage[T1]{fontenc}
  \usepackage[utf8]{inputenc}
\else % if luatex or xelatex
  \ifxetex
    \usepackage{mathspec}
  \else
    \usepackage{fontspec}
  \fi
  \defaultfontfeatures{Ligatures=TeX,Scale=MatchLowercase}
\fi
% use upquote if available, for straight quotes in verbatim environments
\IfFileExists{upquote.sty}{\usepackage{upquote}}{}
% use microtype if available
\IfFileExists{microtype.sty}{%
\usepackage{microtype}
\UseMicrotypeSet[protrusion]{basicmath} % disable protrusion for tt fonts
}{}
\usepackage[margin=1in]{geometry}
\usepackage{hyperref}
\hypersetup{unicode=true,
            pdftitle={VC theory},
            pdfborder={0 0 0},
            breaklinks=true}
\urlstyle{same}  % don't use monospace font for urls
\usepackage{graphicx,grffile}
\makeatletter
\def\maxwidth{\ifdim\Gin@nat@width>\linewidth\linewidth\else\Gin@nat@width\fi}
\def\maxheight{\ifdim\Gin@nat@height>\textheight\textheight\else\Gin@nat@height\fi}
\makeatother
% Scale images if necessary, so that they will not overflow the page
% margins by default, and it is still possible to overwrite the defaults
% using explicit options in \includegraphics[width, height, ...]{}
\setkeys{Gin}{width=\maxwidth,height=\maxheight,keepaspectratio}
\IfFileExists{parskip.sty}{%
\usepackage{parskip}
}{% else
\setlength{\parindent}{0pt}
\setlength{\parskip}{6pt plus 2pt minus 1pt}
}
\setlength{\emergencystretch}{3em}  % prevent overfull lines
\providecommand{\tightlist}{%
  \setlength{\itemsep}{0pt}\setlength{\parskip}{0pt}}
\setcounter{secnumdepth}{0}
% Redefines (sub)paragraphs to behave more like sections
\ifx\paragraph\undefined\else
\let\oldparagraph\paragraph
\renewcommand{\paragraph}[1]{\oldparagraph{#1}\mbox{}}
\fi
\ifx\subparagraph\undefined\else
\let\oldsubparagraph\subparagraph
\renewcommand{\subparagraph}[1]{\oldsubparagraph{#1}\mbox{}}
\fi

%%% Use protect on footnotes to avoid problems with footnotes in titles
\let\rmarkdownfootnote\footnote%
\def\footnote{\protect\rmarkdownfootnote}

%%% Change title format to be more compact
\usepackage{titling}

% Create subtitle command for use in maketitle
\newcommand{\subtitle}[1]{
  \posttitle{
    \begin{center}\large#1\end{center}
    }
}

\setlength{\droptitle}{-2em}
  \title{VC theory}
  \pretitle{\vspace{\droptitle}\centering\huge}
  \posttitle{\par}
  \author{}
  \preauthor{}\postauthor{}
  \predate{\centering\large\emph}
  \postdate{\par}
  \date{2019-08-03}


\begin{document}
\maketitle

\begin{quote}
Note: most of the material in this post follows the presentation in
\emph{Understanding Machine Learning: From Theory to Algorithms} by
Shalev-Shwartz and Ben-David
\end{quote}

As usual, we start with a training set of labeled examples \(S\) \[
(X_1, Y_1)\;, \;\ldots\;,\; (X_n, Y_n) \overset{\text{iid}}{\sim} P
\] from some joint distribution over \(\mathcal{X}\times \mathcal{Y}\),
and our goal is to output a predictive function
\(h: \mathcal{X} \rightarrow \mathcal{Y}\) that has low
\emph{generalization error} or \emph{risk} \[
L_P(h) = \underset{{(X,Y)\sim P}}{\mathbf{E}}\left[ \ell(h, (X,Y))\right]\;.
\] For binary classification problems with \(\mathcal{Y} = \{0, 1\}\),
we often employ the 0-1 loss
\(\ell_{01}(h, (x,y)) = \mathbf{1}(h(x)\ne y)\) which yields the risk
\(L_P(h) = \mathbb{P}\left\{ h(X) \ne Y \right\}\). This is
statistically intuitive, but for computational reasons (which we don't
consider in this post), it might make sense to think about different
losses.

Since we don't know \(P\), we can't compute the risk. We can approximate
it with the \emph{empirical risk}, defined as \[
L_S(h) = \frac{1}{n}\sum_{i=1}^n \ell(h, (X_i, Y_i))
\] which in the binary classification 0-1 loss case, equals
\(L_S(h) = \frac{1}{n} \sum_{i=1}^n \mathbf{1}(h(X_i) \ne Y_i)\) and
records the fraction of mistakes that \(h\) makes on the training set.
If we limit our consideration to some hypothesis class of functions
\(\mathcal{H}\) that represents our prior knowledge, and we let \(h^*\)
denote the minimizer of the risk \(L_P(h)\) over \(\mathcal{H}\), the
following definition encapsulates a natural notion of learnability.

\begin{center}\rule{0.5\linewidth}{\linethickness}\end{center}

\textbf{Definition 1:} A hypothesis class \(\mathcal{H}\) is
\emph{agnostic (proper) PAC learnable} if there exists a learning
algorithm
\(\hat{h}:(\mathcal{X}\times\mathcal{Y})^n \rightarrow \mathcal{H}\)
that satisfies the following property for every
\(\epsilon, \delta \in (0,1)\) and every source distribution \(P\), once
presented with enough samples
\(n \ge n(\mathcal{H}, \epsilon, \delta)\): with probability at least
\(1 - \delta\) over training sets of size \(n\) drawn iid \(S\sim P^n\)
\[
L_P(\hat{h}(S)) \le L_P(h^*) + \epsilon\;.
\]

\begin{center}\rule{0.5\linewidth}{\linethickness}\end{center}

In effect, all this is saying is that it is desirable to have
consistency in the sense that \(L_P(\hat{h}(S)) \rightarrow L_P(h^*)\)
in probability. Since \(h^*\) minimizes the risk, it makes sense to take
\(\hat{h}\) to minimize the empirical risk. This turns out to be the
only learning rule of statistical relevance, and it is called the
\emph{ERM learning rule}: \[
\hat{h}(S) = \arg\min_{h\in \mathcal{H}}\; L_S(h)\;.
\] So for large enough sample size \(n\), we want
\(L_S(\hat{h}(S)) \approx L_P(h^*)\). For any fixed classifier
\(h \in \mathcal{H}\), this is simply a consequence of LLN, but since
\(\hat{h}(S)\) is random and depends on the training data, we need a
stronger result than LLN. The cartoon graphic below (recreated from the
\emph{Generalization I} lecture given at Simons Institute) illustrates
the situtation.

\begin{figure}[htbp]
\centering
\includegraphics{/images/empirical_process_1.jpg}
\caption{}
\end{figure}

Broadly speaking, we want \(\mathcal{H}\) to be large enough to ``get
close'' to capturing the best function \(h^* \approx \tilde{h}\)
(i.e.~low \emph{bias}), but we also want \(\mathcal{H}\) to be small
enough so that our \(\hat{h}(S)\) obtained from the sample ``curve''
\(L_S(h)\) is a good approximation \(\hat{h}(S) \approx h^*\) (i.e.~low
\emph{variance}). Clearly the second criterion will be satisfied if the
sampled curve \(L_S(h)\) is uniformly close to \(L_P(h)\) for all
\(h\in \mathcal{H}\) with high probability. This leads us to another
definition.

\begin{center}\rule{0.5\linewidth}{\linethickness}\end{center}

\textbf{Definition 2:} A hypothesis class \(\mathcal{H}\) has the
\emph{uniform convergence property} (and is called a
\emph{Glivenko-Cantelli class}) if it satisfies the following property
for every \(\epsilon, \delta \in (0,1)\) and every source distribution
\(P\), once given enough training samples
\(n\ge n^{\text{uc}}(\mathcal{H}, \epsilon, \delta)\): with probability
at least \(1 - \delta\) over training sets of size \(n\) drawn iid
\(S\sim P^n\) \[
\forall h\in \mathcal{H},  \quad |L_S(h) - L_P(h)| \le \epsilon\;.
\]

\begin{center}\rule{0.5\linewidth}{\linethickness}\end{center}

It's not hard to show that if \(\mathcal{H}\) has the uniform
convergence property with sample complexity
\(n^{\text{uc}}(\mathcal{H}, \epsilon, \delta)\), then \(\mathcal{H}\)
is agnostic PAC learnable (via the ERM learning rule) with sample
complexity
\(n(\mathcal{H}, \epsilon, \delta) \le n^{\text{uc}}(\mathcal{H}, \epsilon/2, \delta)\).
This can be seen as follows: take a fixed sample
\(S\in (\mathcal{X}\times\mathcal{Y})^n\) that falls in the ``uniform
convergence'' event above for parameter values \(\epsilon/2, \delta\).
For such an \(S\), we have that for all \(h\in \mathcal{H}\) \[
\begin{align*}
L_P(\hat{h}(S)) \le L_S(\hat{h}(S)) + \frac{\epsilon}{2} \le L_S(h) + \frac{\epsilon}{2} \le L_P(h) + \frac{\epsilon}{2} + \frac{\epsilon}{2} = L_P(h) + \epsilon
\end{align*}
\] where the first and third inequalities rely on the fact that \(S\) is
in the uniform convergence event. Therefore, the above statement holds
with probability at least \(1 - \delta\), and since \(h\) is arbitrary,
we can take the infimum on the rhs over \(\mathcal{H}\) to recover the
agnostic PAC learnability requirement.

So possession of the uniform convergence property is sufficient for
\(\mathcal{H}\) to be agnostic PAC learnable. We will now describe a
sufficient condition for \(\mathcal{H}\) to have the uniform convergence
property in the binary classification case. First we note that the
uniform convergence property can be expressed as follows: for every
\(\epsilon, \delta \in (0,1)\) and every source distribution \(P\), once
given enough training samples
\(n\ge n^{\text{uc}}(\mathcal{H}, \epsilon, \delta)\) \[
\underset{S\sim P^n}{\mathbb{P}} \left\{ \sup_{h\in \mathcal{H}} |L_S(h) - L_P(h)| > \epsilon\right\} < \delta
\] so it is logical to pursue bounds of the form \[
\underset{S\sim P^n}{\mathbb{P}} \left\{ \sup_{h\in \mathcal{H}} |L_S(h) - L_P(h)| > \epsilon\right\} < \underbrace{ \binom{\text{function class}}{\text{complexity of } \mathcal{H}} \cdot \binom{\text{concentration bound}}{\text{for any fixed } h}}_{\text{goes to 0 as } n\rightarrow \infty \text{ with fixed }\epsilon} \tag{*}
\] since then we can set the rhs equal to \(\delta\) and solve for the
appropriate sample size \(n\). Recall that Hoeffding's inequality can be
applied for a fixed \(h\in \mathcal{H}\) to obtain the concentration
bound \[
\underset{S\sim P^n}{\mathbb{P}}\{|L_S(h) - L_P(h)| > \epsilon\} \le 2e^{-2n\epsilon^2}
\] which is exponentially decreasing in \(n\); the key is that we need
our notion of function class complexity of \(\mathcal{H}\) to grow
slowly enough with \(n\) to be killed off by the concentration bound in
the limit. Now we are ready to introduce more definitions.

\begin{center}\rule{0.5\linewidth}{\linethickness}\end{center}

\textbf{Definition 3:} The \emph{growth function} of a hypothesis class
\(\mathcal{H}\) is defined as \[
\Gamma(\mathcal{H}, n) = \sup_{\{x_1, \ldots, x_n\} \subseteq \mathcal{X}} \left| \mathcal{H}_{x_1, \ldots, x_n} \right|
\] where \[
\mathcal{H}_{x_1, \ldots, x_n} = \left\{(h(x_1), \ldots, h(x_n)): h\in \mathcal{H} \right\}
\] is the set of all possible \emph{behaviors} (or \emph{realizations})
of functions in \(\mathcal{H}\) on the set of points
\(\{x_1, \ldots, x_n\}\). Since a class of binary functions
\(\mathcal{H}\) can be identified with the collection
\(\{h^{-1}(1): h\in \mathcal{H}\}\), the growth function of
\(\mathcal{H}\) \[
\Gamma(\mathcal{H}, n) = \sup_{C\subseteq \mathcal{X},\; |C|=n} |\{h^{-1}(1)\cap C: h\in \mathcal{H}\}
\] counts the number of subsets of \(C=\{x_1, \ldots, x_n\}\) that get
``picked out'' by \(\mathcal{H}\). The \emph{VC dimension} of a class of
binary functions \(\mathcal{H}\) is defined as \[
\mathsf{VC}(\mathcal{H}) = \sup \{n: \Gamma(\mathcal{H}, n) = 2^n\}
\] or in words, the maximum \(n\) such that there exists a set
\(C\subseteq \mathcal{X}\) of size \(n\) on which \(\mathcal{H}\) picks
out all possible subsets; such a set \(C\) is said to be
\emph{shattered} by \(\mathcal{H}\).

\begin{center}\rule{0.5\linewidth}{\linethickness}\end{center}

It turns out that we can prove bounds of the form \((^*)\) using the
growth function \(\Gamma(\mathcal{H}, n)\) as the function class
complexity. What makes these bounds useful is the surprising fact that
if \(\mathsf{VC}(\mathcal{H}) < \infty\), the growth function grows only
polynomially for \(n \ge \mathsf{VC}(\mathcal{H})\). We state this
famous result below without proof.

\begin{center}\rule{0.5\linewidth}{\linethickness}\end{center}

\textbf{Theorem (Sauer's Lemma):} If
\(d\doteq \mathsf{VC}(\mathcal{H}) < \infty\) then \[
\Gamma(\mathcal{H}, n) \le \sum_{i=0}^d \binom{n}{i} \le \left(\frac{en}{d}\right)^d
\] where the second inequality holds for \(n\ge d\).

\begin{center}\rule{0.5\linewidth}{\linethickness}\end{center}

With this fact in hand, we now work to establish uniform bounds of the
form \((^*)\). Notice that the growth function is a combinatorial
measure of complexity; it characterizes \(\mathcal{H}\) in terms of its
action on finite sets of points. The technique of symmetrization allows
us to massage the lhs of \((^*)\) into a form amenable for bounding by
the growth function.

\begin{center}\rule{0.5\linewidth}{\linethickness}\end{center}

\textbf{Lemma (Symmetrization):} For any hypothesis class
\(\mathcal{H}\), source distribution \(P\), \(\epsilon > 0\), and sample
size \(n > \frac{2}{\epsilon^2}\): \[
\underset{S\sim P^n}{\mathbb{P}} \left\{ \sup_{h\in \mathcal{H}} |L_S(h) - L_P(h)| > \epsilon\right\} \le 2\, 
\underset{S,S'\sim P^n}{\mathbb{P}} \left\{ \sup_{h\in \mathcal{H}} |L_S(h) - L_{S'}(h)| > \frac{\epsilon}{2}\right\}
\] where \(S, S'\) are iid training sets (\(S'\) is called a \emph{ghost
sample}).

\begin{center}\rule{0.5\linewidth}{\linethickness}\end{center}

\begin{quote}
\emph{Proof.} For simplicity, we will assume that \[
h_S = \arg\sup_{h\in \mathcal{H}}\; |L_S(h) - L_P(h)|
\] is realized. Then we make the claim \[
\mathbf{1}\left\{ |L_S(h_S) - L_P(h_S)| > \epsilon\right\} \cdot \mathbf{1}\left\{ |L_{S'}(h_S) - L_{P}(h_S)| \le \frac{\epsilon}{2}\right\} \le \mathbf{1}\left\{  |L_S(h_S) - L_{S'}(h_S)| > \frac{\epsilon}{2}\right\} \tag{*}
\] which follows from a simple triangle inequality argument: \[
\begin{align*}
\epsilon &<  |L_S(h_S) - L_P(h_S)| \\[5px]
&\le |L_S(h_S) - L_{S'}(h_S)| + |L_{S'}(h_S) - L_P(h_S)| \\[5px]
&\le  |L_S(h_S) - L_{S'}(h_S)| + \frac{\epsilon}{2}\;.
\end{align*}
\] Now we note that \[
\begin{align*}
\underset{S'\sim P^n}{\mathbb{P}}\left\{ |L_{S'}(h_S) - L_{P}(h_S)| > \frac{\epsilon}{2}\right\}
&\le \frac{4}{\epsilon^2}\cdot \text{Var}\left[ L_{S'}(h_S)\right] \\[5px]
&= \frac{4}{n\epsilon^2}\cdot \text{Var}\big[\mathbf{1}\{h_S(X)\ne Y\}\big] \\[5px] 
&\le \frac{1}{n\epsilon^2}
\end{align*}
\] where we used the fact that the maximum variance of a random variable
taking values in \([0,1]\) is \(\frac{1}{4}\) (realized by a fair coin
flip), which implies that for \(n > \frac{2}{\epsilon^2}\) \[
\underset{S'\sim P^n}{\mathbb{P}}\left\{ |L_{S'}(h_S) - L_{P}(h_S)| \le \frac{\epsilon}{2}\right\} > \frac{1}{2}\;.
\] So taking expectations on both sides of \((^*)\) over \(S'\) for
\(n > \frac{2}{\epsilon^2}\) gives \[
\mathbf{1}\left\{ |L_S(h_S) - L_P(h_S)| > \epsilon\right\} \le 2 \cdot \underset{S'\sim P^n}{\mathbb{P}}\left\{ |L_S(h_S) - L_{S'}(h_S)| > \frac{\epsilon}{2}\right\}
\] which implies that \[
\mathbf{1}\left\{ \sup_{h\in \mathcal{H}} |L_S(h) - L_P(h)| > \epsilon\right\} \le 2 \cdot \underset{S'\sim P^n}{\mathbb{P}}\left\{ \sup_{h\in \mathcal{H}} |L_S(h) - L_{S'}(h)| > \frac{\epsilon}{2}\right\}
\] and finally taking expectation over \(S\) above gives the desired
result.
\end{quote}

\begin{center}\rule{0.5\linewidth}{\linethickness}\end{center}

Symmetrization allows us to restrict our consideration to the behavior
of each \(h\) on the finite samples \(S, S'\). The above lemma implies
that \[
\begin{align*}
\underset{S\sim P^n}{\mathbb{P}} \left\{ \sup_{h\in \mathcal{H}} |L_S(h) - L_P(h)| > \epsilon\right\} &\le 2\cdot \underset{S,S'\sim P^n}{\mathbb{P}} \left\{ \sup_{h\in \mathcal{H}} |L_S(h) - L_{S'}(h)| > \frac{\epsilon}{2}\right\} \\[5px]
&= 2\cdot \underset{S,S'\sim P^n}{\mathbb{P}} \left\{ \max_{v\in \mathcal{H}_C} \frac{1}{n}\left|\sum_{i=1}^n \mathbf{1}\{v(X_i)\ne Y_i\} - \mathbf{1}\{v(X_i')\ne Y_i'\}\right| > \frac{\epsilon}{2}\right\}\\[5px]
\end{align*}
\] where \(C = \{ X_1, \ldots, X_n, X_1', \ldots, X_n'\}\) is a set of
\(2n\) examples from \(\mathcal{X}\). We would like to union bound over
\(\mathcal{H}_C\) and bound the argument using Hoeffding's inequality,
but the problem is that \(\mathcal{H}_C\) is a random collection and
can't be pulled out of the probability expression directly. If we
condition on \(S, S'\), \(\mathcal{H}_C\) is a fixed collection and we
can use the union bound, but then there is no more randomness left to
apply Hoeffding's inequality. So somewhat unintuitively, the trick is to
introduce \emph{more randomness} in a clever way. First, we observe that
if \(Z, Z'\) are iid random variables and
\(\sigma\sim \text{unif}\{\pm 1\}\) is an independent (fair) coin flip,
then \[
Z - Z' \overset{d}{=} \sigma(Z-Z')
\] since \[
\mathbb{P}\{ \sigma(Z - Z') \le t\} = \frac{1}{2}\cdot \mathbb{P}\{Z - Z' \le t\} + \frac{1}{2}\cdot \mathbb{P} \{Z' - Z \le t\} = \mathbb{P}\{Z - Z'\le t\}\;.
\] It then follows that \[
\max_{v\in \mathcal{H}_C} \frac{1}{n}\left|\sum_{i=1}^n \mathbf{1}\{v(X_i)\ne Y_i\} - \mathbf{1}\{v(X_i')\ne Y_i'\}\right| \overset{d}{=} \max_{v\in \mathcal{H}_C} \frac{1}{n}\left|\sum_{i=1}^n \sigma_i(\mathbf{1}\{v(X_i)\ne Y_i\} - \mathbf{1}\{v(X_i')\ne Y_i'\})\right|
\] so continuing our calculation from before, \[
\begin{align*}
\underset{S\sim P^n}{\mathbb{P}} \left\{ \sup_{h\in \mathcal{H}} |L_S(h) - L_P(h)| > \epsilon\right\} &\le 2\cdot \underset{S,S'\sim P^n}{\mathbb{P}} \left\{ \max_{v\in \mathcal{H}_C} \frac{1}{n}\left|\sum_{i=1}^n \mathbf{1}\{v(X_i)\ne Y_i\} - \mathbf{1}\{v(X_i')\ne Y_i'\}\right| > \frac{\epsilon}{2}\right\}\\[5px]
&= 2\cdot\underset{S,S'\sim P^n, \sigma\sim \text{unif}\{\pm 1\}^n}{\mathbb{P}}\left\{ \max_{v\in \mathcal{H}_C} \frac{1}{n}\left|\sum_{i=1}^n \sigma_i(\mathbf{1}\{v(X_i)\ne Y_i\} - \mathbf{1}\{v(X_i')\ne Y_i'\})\right| > \frac{\epsilon}{2}\right\}\\[5px]
&= 2\cdot\underset{S,S'\sim P^n}{\mathbf{E}} \left[ \underset{\sigma\sim \text{unif}\{\pm 1\}^n}{\mathbb{P}}\left\{ \max_{v\in \mathcal{H}_C} \frac{1}{n}\left|\sum_{i=1}^n \sigma_i(\mathbf{1}\{v(X_i)\ne Y_i\} - \mathbf{1}\{v(X_i')\ne Y_i'\})\right| > \frac{\epsilon}{2}\right\}\right] \\[5px]
&\le 2\cdot\underset{S,S'\sim P^n}{\mathbf{E}} \left[ \sum_{v\in \mathcal{H}_C} \underset{\sigma\sim \text{unif}\{\pm 1\}^n}{\mathbb{P}}\left\{ \frac{1}{n}\left|\sum_{i=1}^n \sigma_i(\mathbf{1}\{v(X_i)\ne Y_i\} - \mathbf{1}\{v(X_i')\ne Y_i'\})\right| > \frac{\epsilon}{2}\right\}\right] \\[5px]
&\le 2\cdot\underset{S,S'\sim P^n}{\mathbf{E}} \left[ 2|\mathcal{H}_C|e^{-n\epsilon^2/2}\right] \\[5px]
&\le 4\cdot \Gamma(\mathcal{H}, 2n)\,e^{-n\epsilon^2/2}
\end{align*}
\] and if \(\mathcal{H}\) has finite VC-dimension \(d\), then Sauer's
lemma implies that for \(n\ge d\) (and \(n > \frac{2}{\epsilon^2}\) for
symmetrization) we have \[
\underset{S\sim P^n}{\mathbb{P}} \left\{ \sup_{h\in \mathcal{H}} |L_S(h) - L_P(h)| > \epsilon\right\} \le 4\left( \frac{2en}{d}\right)^d e^{-n\epsilon^2/2}\;.
\] and further algebraic manipulation (by setting the rhs equal to
\(\delta\)) yields a sufficient sample complexity
\(n^{\text{uc}}(\mathcal{H}, \epsilon, \delta)\) in terms of \(d\). We
note that it is possible to use more sophisticated methods and obtain a
tighter sample complexity.

So finite VC-dimension implies \(\mathcal{H}\) has the uniform
convergence property, which in turn implies agnostic PAC learnability.
It turns out that (in the binary classification setting) finite
VC-dimension is also a necessary condition for agnostic PAC
learnability. The following result shows that if
\(\mathsf{VC}(\mathcal{H}) = \infty\) then the function class is too
rich and admits adversarial source distributions that prevent
learnability.

\begin{center}\rule{0.5\linewidth}{\linethickness}\end{center}

\textbf{Theorem (No-Free-Lunch):} Let
\(\hat{h}:(\mathcal{X}\times\mathcal{Y})^n \rightarrow \mathcal{H}\) be
any learning algorithm for the task of binary classification with
respect to the 0-1 loss. Let \(n\) be any sample size, and assume that
there exists a subset \(C\subseteq \mathcal{X}\) of size \(2n\) that is
shattered by \(\mathcal{X}\). Then there exists a distribution \(P\)
over \(\mathcal{X}\times\{0,1\}\) such that:

\begin{enumerate}
\def\labelenumi{\arabic{enumi}.}
\item
  There exists a function \(h^*\in \mathcal{H}\) with \(L_P(h^*) = 0\).
\item
  With probability at least \(\frac{1}{7}\) over training sets of size
  \(n\) drawn iid \(S\sim P^n\) we have that
  \(L_P(\hat{h}(S)) \ge \frac{1}{8}\).
\end{enumerate}

We conclude that when \(\mathcal{H}\) has infinite VC-dimension it is
not PAC learnable by taking \(\epsilon < \frac{1}{8}\) and
\(\delta < \frac{1}{7}\).

\begin{center}\rule{0.5\linewidth}{\linethickness}\end{center}

\begin{quote}
\emph{Proof:} Note that \(|\mathcal{H}_C| = 2^{2n}\) and we enumerate
these binary labellings as \(v_1, \ldots, v_m\). For each \(v_i\) define
the corresponding distribution \(P_i\) to be uniform over the \(2n\)
pairs \((x, v_i(x))\) for \(x\in C\) (i.e.~we pick \(x\) from \(C\)
uniformly at random and deterministically set the label to be
\(v_i(x)\)), and let \(h_i \in \mathcal{H}\) be a function whose
projection onto \(C\) is \(v_i\). Clearly \(L_{P_i}(h_i) = 0\). Let
\(V\in \{v_1, \ldots, v_m\}\) be a random labelling drawn according to
the uniform distribution (i.e.~we set each component of \(V\) to be 0 or
1 via an independent fair coin toss). We will show that \[
\underset{V}{\mathbf{E}}\;\underset{S\sim P_{_V}^n}{\mathbf{E}}\left[ L_{P_{_V}}(\hat{h}(S))\right] \ge \frac{1}{4} \tag{*}
\] which implies that for at least one \(v_i\) labelling vector
\(\mathbf{E}_{_{S\sim P_{i}^n}}\left[ L_{P_{i}}(\hat{h}(S))\right] \ge \frac{1}{4}\)
and since \[
\begin{align*}
\frac{1}{4} &\le \underset{S\sim P_{i}^n}{\mathbf{E}}\left[ L_{P_{i}}(\hat{h}(S))\right] \\[5px]
&= \underset{S\sim P_{i}^n}{\mathbf{E}}\left[ L_{P_{i}}(\hat{h}(S)) \cdot \mathbf{1}\left\{L_{P_{i}}(\hat{h}(S)) \ge \frac{1}{8}\right\}\right] + \underset{S\sim P_{i}^n}{\mathbf{E}}\left[ L_{P_{i}}(\hat{h}(S)) \cdot \mathbf{1}\left\{L_{P_{i}}(\hat{h}(S)) < \frac{1}{8}\right\}\right] \\[5px]
&\le \underset{S\sim P_{i}^n}{\mathbb{P}}\left\{L_{P_{i}}(\hat{h}(S)) \ge \frac{1}{8}\right\} + \frac{1}{8} \left(1 - \underset{S\sim P_{i}^n}{\mathbb{P}}\left\{L_{P_{i}}(\hat{h}(S)) \ge \frac{1}{8}\right\} \right) \\[5px]
&= \frac{1}{8} + \frac{7}{8}\cdot\underset{S\sim P_{i}^n}{\mathbb{P}}\left\{L_{P_{i}}(\hat{h}(S)) \ge \frac{1}{8}\right\}
\end{align*}
\] we get the desired conclusion. Now returning to verifying \((^*)\),
we define the following terms: a sample \(s\) is \emph{consistent} with
the label vector \(v\) if \(s\) has nonzero probability of being drawn
from \(P_v^n\), and a sample \(s\) is \emph{self-consistent} if it is
consistent with any \(v\in \mathcal{H}_C\). Next, we perform the
following clever manipulation: \[
\begin{align*}
\underset{V}{\mathbf{E}}\;\underset{S\sim P_{_V}^n}{\mathbf{E}}\left[ L_{P_{_V}}(\hat{h}(S))\right] &= \underset{V}{\mathbf{E}}\;\underset{S\sim P_{_V}^n}{\mathbf{E}}\left[ L_{P_{_V}}(\hat{h}(S)) \; \middle| \; S \text{ cons. with } V \right] \\[5px]
&= \underset{V}{\mathbf{E}}\;\underset{S\sim \text{ self-cons.}}{\mathbf{E}}\left[ L_{P_{_V}}(\hat{h}(S)) \; \middle| \; S \text{ cons. with } V \right] \\[5px]
&= \underset{S\sim \text{ self-cons.}}{\mathbf{E}}\;\underset{V}{\mathbf{E}}\left[ L_{P_{_V}}(\hat{h}(S)) \; \middle| \; S \text{ cons. with } V \right] \\[5px]
&= \underset{S\sim \text{ self-cons.}}{\mathbf{E}}\;\underset{V}{\mathbf{E}}\left[ \frac{1}{|C|}\sum_{x\in C} \mathbf{1}\left\{ \hat{h}(S)(x) \ne V(x)\right\}  \; \middle| \; S \text{ cons. with } V \right] \\[5px]
&= \underset{S\sim \text{ self-cons.}}{\mathbf{E}}\;\underset{V}{\mathbf{E}}\left[ \frac{1}{2n}\left(\sum_{x\in C, \,x\in S} \mathbf{1}\left\{ \hat{h}(S)(x) \ne V(x)\right\} + \sum_{x\in C,\,x\not\in S}\mathbf{1}\left\{ \hat{h}(S)(x) \ne V(x)\right\} \right)\; \middle| \; S \text{ cons. with } V \right] \\[5px]
&= \underset{S\sim \text{ self-cons.}}{\mathbf{E}}\left[ \frac{1}{2n}\sum_{x\in C,\,(x,y)\in S} \mathbf{1}\left\{ \hat{h}(S)(x) \ne y\right\}\right] + \underset{S\sim \text{ self-cons.}}{\mathbf{E}}\;\underset{V}{\mathbf{E}}\left[\frac{1}{2n}\sum_{x\in C, \,x\not\in S}\mathbf{1}\left\{ \hat{h}(S)(x) \ne V(x)\right\}\; \middle| \; S \text{ cons. with } V \right] \\[5px]
&= \underset{S\sim \text{ self-cons.}}{\mathbf{E}}\left[ \frac{1}{2n}\sum_{x\in C,\,(x,y)\in S} \mathbf{1}\left\{ \hat{h}(S)(x) \ne y\right\}\right] + \underset{S\sim \text{ self-cons.}}{\mathbf{E}}\left[\frac{1}{2n}\cdot\frac{|C\setminus S|}{2}\right] \\[5px]
&\ge 0 + \underset{S\sim \text{ self-cons.}}{\mathbf{E}}\left[\frac{1}{2n}\cdot\frac{n}{2}\right] \\[5px]
&= \frac{1}{4}
\end{align*}
\] where \(S \sim \text{ self-cons.}\) indicates that \(S\) is drawn
uniformly from the set of all self-consistent samples.
\end{quote}

\begin{center}\rule{0.5\linewidth}{\linethickness}\end{center}

It is worth distilling the main ideas at play in the proof above. Notice
that uniform distributions are ubiquitous and play a key role; the
conditioning in the first step allows us to take \(S\) to be uniform
over the larger set of self-consistent samples, since the conditioning
``projects'' \(S\) back to being uniform over samples that are
consistent with \(V\). This notational trick enables us to use Fubini's
theorem to swap the order of expectations. So now the randomness in the
conditional is with respect to \(V\) first. Then we split our
consideration between those \(x\in C\) that appear in \(S\) and those
that do not; the key fact here is that \(|C| = 2n\) while \(|S| = n\),
so fixing \(S\) and enforcing consistency between \(S\) and \(V\) still
allows for at least half of the ``bits'' in \(V\) to be randomly set.
This freedom stems from the fact that the set \(C\) is shattered by
\(\mathcal{H}\), so all possible binary labellings are realizable.

So finally, putting all of these facts together, we have established the
following remarkable result.

\begin{center}\rule{0.5\linewidth}{\linethickness}\end{center}

\textbf{Theorem (The Fundamental Theorem of Statistical Learning):} Let
\(\mathcal{H}\) be a hypothesis class of functions from a domain
\(\mathcal{X}\) to \(\{0, 1\}\) and let the loss function be the 0-1
loss. Then the following are equivalent:

\begin{enumerate}
\def\labelenumi{\arabic{enumi}.}
\item
  \(\mathcal{H}\) has finite VC-dimension.
\item
  \(\mathcal{H}\) has the uniform convergence property.
\item
  \(\mathcal{H}\) is agnostic PAC learnable (via the ERM rule).
\end{enumerate}

\begin{center}\rule{0.5\linewidth}{\linethickness}\end{center}


\end{document}
